\documentclass[10pt]{article}
\setlength{\parindent}{0pt}
% \usepackage{setspace} \doublespacing
\usepackage[a4paper, total={6.3in, 9in}]{geometry}
\usepackage{graphicx}
\usepackage{listings}
\usepackage{xcolor}
\usepackage{pdfpages}

% A custom title generation command for physics labs.
\newcommand{\PLtitle}{\setlength{\parindent}{0pt}
\begin{center}

  \huge{\textbf{Lab \Lab: \Ltitle\\}}
  \normalsize 

  Physics 1320L-02 \\
  TA: \TA \\
  Date: \date\\
  Name: \author\\
  
  Team: \team\\


\end{center}
}


\definecolor{codegreen}{rgb}{0,0.6,0}
\definecolor{codegray}{rgb}{0.5,0.5,0.5}
\definecolor{codepurple}{rgb}{0.58,0,0.82}
\definecolor{backcolour}{rgb}{0.95,0.95,0.92}

\lstdefinestyle{mystyle}{
    backgroundcolor=\color{backcolour},   
    commentstyle=\color{codegreen},
    keywordstyle=\color{magenta},
    numberstyle=\tiny\color{codegray},
    stringstyle=\color{codepurple},
    basicstyle=\ttfamily\footnotesize,
    breakatwhitespace=false,         
    breaklines=true,                 
    captionpos=b,                    
    keepspaces=true,                 
    numbers=left,                    
    numbersep=5pt,                  
    showspaces=false,                
    showstringspaces=false,
    showtabs=false,                  
    tabsize=2
}

\lstset{style=mystyle}

% Defining variables for PLtitle
\def\Lab{6}
\def\Ltitle{Capacitors}
\def\author{Kassidy Maberry}
\def\team{Lukas Chavez}
\def\TA{Jonathan Dooley}
\def\date{2024/03/25}
% Defining a variable for creating tabs since writing \hspace*{10mm} is tedious.
\def\tab{\hspace*{10mm}}
\def\halftab{\hspace*{5mm}}

\begin{document}
% Generate title
\PLtitle

% --- Introduction --- % 
\section{Introduction}
\tab Capacitors are devices that are capable of storing electric charge. This is done with two metal plates that
are insulated from eacother storing the charge. One plate will store a positive charge and the other will store a negative charge.
They will continue to hold this charge until the source of electricity is removed. Then the charges will begin to interact with each other
through the insulator.


% --- Methods --- %
\section{Methods}


% --- Data --- %
\section{Data}

% --- Analysis --- %
\section{Analysis}
\subsection*{Effective Capacitance}

%1. Calculate and record your theoretical effective series capacitance and the theoretical effective parallel
%capacitance for the capacitors used.

%2. Compare your experimental values to your theoretical values and obtain the percentage difference
% between both.

\subsection*{Parallel Plate Capacitor}

% 1.Plot the capacitance of acrylic, Cacrylic, versus the capacitance of air, Cair.

% 2.Use this plot and Equation 4 to determine the dielectric constant, κ, for acrylic. (Try to figure it
% out yourself first, then ask your instructor as needed.) Be sure to explain your work.

\subsection*{Cylindrical Capacitor}

% 1.Find the theoretical value for the capacitance of the cylindrical capacitor

% 2.Compare the theoretical and experimental values of capacitance for the cylindrical capacitor using
% percent error.


% 3.Use Gauss's Law to outline the derivation of the concentric cylinder capacitor without dielectric
% eq(5).


% --- Conclusion --- %
\section{Discussion and Conclusion}


% --- Appendix
\section*{Appendix}\newpage
\includepdf[pages={1-3}]{Data.pdf}


% Insert PDF
\end{document}